\documentclass[a4paper,12pt]{ctexart}
\usepackage[utf8]{inputenc}
\usepackage{graphicx}
\usepackage{amsmath}
\usepackage{hyperref}
\usepackage{bookmark}
\hypersetup{
    hidelinks
}
\title{五子棋项目报告}
\author{潘宇轩}
\date{\today}

\begin{document}

\maketitle

\begin{abstract}
    采用搜索算法完成五子棋的人机对战
\end{abstract}

\tableofcontents
\section{第一版}

\subsection{搜索方法}

minimax算法, alpha-beta剪枝

\subsection{可视化}
使用SDL2库,实现棋盘、棋子可视化.修改颜色使得接近真实五子棋的颜色

\subsection{分析}
可以初步五子棋的人机对战,但是效率不高。仅能搜索2层。

\subsection{计划}
减少搜索的棋子数量,提高搜索层数。

\section{第二版}

\subsection{裁判功能}
完善裁判功能,添加禁手规则

\subsection{禁手}
采用递归判断,直接根据棋形判断禁手

\subsection{可视化}
添加棋盘标号
\end{document}